\documentclass[../main.tex]{subfiles}



%% Local Macros and packages: add any of your own definitions here.

\begin{document}

% Homework number, your name and NetID, and optionally, the names/NetIDs of anyone you collaborated with. If you did not collaborate with anybody, leave the last parameter empty.
\homework
    {2}
    {Aahan Agrawal (agrawl10)}
    {Some Person (sperson2), Another Person (aperson3)}

\section*{Problem (2)}
(a) \\

Since $A$ is symmetric, $A$ is unitarily diagonalizable, meaning that

\begin{align*}
    A = P^T D P \\
    \intertext{Where $P^T = \inv{P}$ and $D$ is diagonal} \\
    \intertext{Since $D$ is diagonal and $A$ positive semi-definite, the eigenvalues
    of $A$ occupy the diagonal entries of $D$ and all eigenvalues are non-negative. 
As a consequence, we can define a square root for $D$, which is the matrix obtained by taking the square root of each diagonal entry in $D$. We call this matrix $E$}
A = P^T E E P \\
\implies x^TAx' = x^TP^TE EPx' \\
= (EPx)^T(EPx') \\
\intertext{Thus, we see that a feature transformation $\phi$ exists defined by $\phi(x) = EPx$ such that $x^TAx' = k(x,x')  \phi(x)^T\phi(x')$.} \\
\end{align*}

(b) \\

Since $k$ is a valid kernel, $k(x,x')$ can be decomposed into the inner product
of some feature transformation $\phi$. That is, $k(x,x') = \phi(x)^T \phi(x')$. \\

Define a new feature transformation $\psi(x) = f(x) \phi(x)$. Then observe that \\

\[
    \psi(x)^T\psi(x^{\ast}) = f(x)\phi(x)^T\phi(x^{\ast})f(x^{\ast})
\]


(c)

We show that $x^TKx \geq 0$ for all $x \in \R^n$. Recall that inner products produce
non-negative values in $\R$ and that they are symmetric. Thus $K$ is a symmetric matrix
with no negative entries. It follows that 

\begin{align*}
    x^TAx = \sum_{i,j}^{}x_i x_j A_{ij} \\
    = 2 \sum_{i, j > i}^{}x_{i}x_{j} A_{ij}  \\
    \intertext{Since $x_{i}x_{j}A_{ij} = x_{j}x_{i}A_{ij}$} \\
    \intertext{Now suppose that arbitrary $x$ is given. $x$ can be decomposed as the
        sum of two vectors $x_1$ and $x_2$ such that every entry in $x_1$ is non-negative
    and every entry in $x_2$ is non-positive. It follows that} \\
    x^TAx = (x_1 + x_2)^TA(x_1 + x_2) \\
    = x_1^TAx_1 + x_1^TAx_2 + x_2^TAx_2 + x_2^TAx_2 \\
    \intertext{From (1), we know that:} \\
    x_1^TAx_1 = 2 \sum_{i, j > i}^{}x^{i}_{i}x^{i}_{j} A_{ij} \\
    \intertext{From how we defined $x_1$, we conclude that this foregoing expression is non-negative. By similar reasoning, we can conclude that $x_2^TAx_2$ is non-negative} \\
    \intertext{By construction, $x_1^TAx_2$ and $x_2^TAx_1$ are both zero, since wherever $x_1$ is not zero, $x_2$ is zero and vice versa. Hence} \\
    x_1^TAx_1 + x_1^TAx_2 + x_2^TAx_2 + x_2^TAx_2  \geq 0\\
    \intertext{This completes the proof and the problem.}
\end{align*}






\end{document}
