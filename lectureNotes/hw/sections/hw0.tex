\documentclass[../main.tex]{subfiles}



%% Local Macros and packages: add any of your own definitions here.

\begin{document}

% Homework number, your name and NetID, and optionally, the names/NetIDs of anyone you collaborated with. If you did not collaborate with anybody, leave the last parameter empty.
\homework
    {0}
    {Aahan Agrawal (agrawl10)}
    {}

\section*{Problem (1)}
\renewcommand{\l}{\lambda}
\textbf{Solution:}
\pp{1.1} Find those $\l$ so that $\det(A  - \l I) = 0$; then those
$\vec{x}$ such that $(A - \l I)\vec{x} = 0$ (ie the null space of $(A - \l I)$) are the
eigenvectors corresponding to $\l$.

\begin{align*}
    \det \left(
    \begin{bmatrix}
        3 - \l & 1 \\
        8 & 1 - \l \\
    \end{bmatrix} \right) &= 0 \numberthis{eq:det_pol}\\
    (3 - \l)(1 - \l) - 8 &= 0 \\
    3 - 4\l  + \l^2 - 8 &= 0 \\
    \l^2 - 4\l - 5 &= 0 \\
    (\l-5)(\l + 1) &= 0 \\
    \implies \l &= 5, -1 \\
    \intertext{Now let us find the eigenvalue corresponding to $\l = 5$; substitute $\l = 5$ into \eqref{eq:det_pol}} \\
    \begin{bmatrix}
        -2 & 1 \\
        8 & -4 \\
    \end{bmatrix} \\
    \intertext{We see that the null space of this matrix is exactly} \\
    \left\{ \begin{bmatrix}
        1 \\
        2 \\
    \end{bmatrix} \right\} \numberthis{first_vec} \\
    \intertext{Now let us find the eigenvalue corresponding to $\l = -1$; substitute $\l = -1$ into \eqref{eq:det_pol}} \\
    \begin{bmatrix}
        4 & 1 \\
        8 & 2 \\
    \end{bmatrix} \\
    \intertext{We see that the null space of this matrix is exactly} \\
    \left\{ \begin{bmatrix}
        1 \\
        -4 \\
    \end{bmatrix} \right\} \numberthis{second_vec} \\
    \intertext{Thus $\l = 5$ has eigenvector in expression \eqref{first_vec} and $\l = -1$ has eigenvector in expression \eqref{second_vec}} \\
\end{align*}

\pp{1.2}

Suppose that $\l = 0$; then $Ax$ is the zero vector. If $\l > 0$, then $Ax$ points in the direction of $x$; if $\l < 0$ , then $Ax$ points in the direction opposite of $x$. In either case, whether $\l > 0$ or $\l < 0$, the magnitude of $Ax$ is the same as $x$ scaled by $\abs{\l}$.

\pp{1.3}

\begin{proof}
    Suppose that $(\l ,x)$ is an eigenvalue-eigenvector pair of $A$. By definition $Ax = \l x$, whence

    \begin{align*}
        x^TAx &= x^T\l x\\
        &= \l x^T x \\
        &= \l \norm{x}_{2}^{2} \\
        \intertext{Since $x^TAx > 0$, we must have:} \\
        \l \norm{x}_{2}^{2} > 0 \implies \\
        \l &> 0
    \end{align*}
\end{proof}

\section*{Problem (2)}
\textbf{Solution:}
\pp{2.1}
Recall that if $u(x)$ and $v(x)$ are functions of $x$, then 

\[
    
\]<++>






\newpage \nocite{*}
\bibliographystyle{ims}
\bibliography{citations}

\end{document}
